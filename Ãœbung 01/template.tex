\documentclass[a4paper,11pt]{scrartcl}
\usepackage[a4paper, left=2cm, right=4.5cm, top=3.5cm, bottom=2cm]{geometry} % kleinere Ränder

%Paket für Header in Koma-Klassen (scrartcl, scrrprt, scrbook, scrlttr2)
\usepackage[headsepline]{scrlayer-scrpage}
% Header groß genug für 3 Zeilen machen
\setlength{\headheight}{3\baselineskip}


% Default Header löschen
\pagestyle{scrheadings}
\clearpairofpagestyles

% nicht kursiv gedruckte header
\setkomafont{pagehead}{\sffamily\upshape}

% Links im Dokuement sowie \url schön machen
\usepackage[colorlinks,pdfpagelabels,pdfstartview = FitH, bookmarksopen = true,bookmarksnumbered = true, linkcolor = black, plainpages = false, hypertexnames = false, citecolor = black]{hyperref}

% Umlaute in der Datei erlauben, auf deutsch umstellen
\usepackage[utf8]{inputenc}
\usepackage[ngerman]{babel}

% Mathesymbole und Ähnliches
\usepackage{amsmath}
\usepackage{mathtools}
\usepackage{amssymb}
\usepackage{microtype}
\newcommand{\NN}{{\mathbb N}}
\newcommand{\RR}{{\mathbb R}}
\newcommand{\QQ}{{\mathbb Q}}
\newcommand{\ZZ}{{\mathbb{Z}}}

% Komplexitätsklassen
\newcommand{\pc}{\ensuremath{{\sf P}}}
\newcommand{\np}{\ensuremath{{\sf NP}}}
\newcommand{\npc}{\ensuremath{{\sf NPC}}}
\newcommand{\pspace}{\ensuremath{{\sf PSPACE}}}
\newcommand{\exptime}{\ensuremath{{\sf EXPTIME}}}
\newcommand{\CClassNP}{\textup{NP}\xspace}
\newcommand{\CClassP}{\textup{P}\xspace}

% Weitere pakete
\usepackage{multicol}
\usepackage{booktabs}

% Abbildungen
\usepackage{tikz}
\usetikzlibrary{arrows,calc}

% Meistens ist \varphi schöner als \phi, genauso bei \theta
\renewcommand{\phi}{\varphi}
\renewcommand{\theta}{\vartheta}

% Aufzählungen anpassen (alternativ: \arabic, \alph)
\renewcommand{\labelenumi}{(\roman{enumi})}

% rwth colors
% colors: blue violet purple carmine red magenta orange yellow grass cyan gold silver
\definecolor{rwth-blue}{cmyk}{1,.5,0,0}\colorlet{rwth-lblue}{rwth-blue!50}\colorlet{rwth-llblue}{rwth-blue!25}
\definecolor{rwth-violet}{cmyk}{.6,.6,0,0}\colorlet{rwth-lviolet}{rwth-violet!50}\colorlet{rwth-llviolet}{rwth-violet!25}
\definecolor{rwth-purple}{cmyk}{.7,1,.35,.15}\colorlet{rwth-lpurple}{rwth-purple!50}\colorlet{rwth-llpurple}{rwth-purple!25}
\definecolor{rwth-carmine}{cmyk}{.25,1,.7,.2}\colorlet{rwth-lcarmine}{rwth-carmine!50}\colorlet{rwth-llcarmine}{rwth-carmine!25}
\definecolor{rwth-red}{cmyk}{.15,1,1,0}\colorlet{rwth-lred}{rwth-red!50}\colorlet{rwth-llred}{rwth-red!25}
\definecolor{rwth-magenta}{cmyk}{0,1,.25,0}\colorlet{rwth-lmagenta}{rwth-magenta!50}\colorlet{rwth-llmagenta}{rwth-magenta!25}
\definecolor{rwth-orange}{cmyk}{0,.4,1,0}\colorlet{rwth-lorange}{rwth-orange!50}\colorlet{rwth-llorange}{rwth-orange!25}
\definecolor{rwth-yellow}{cmyk}{0,0,1,0}\colorlet{rwth-lyellow}{rwth-yellow!50}\colorlet{rwth-llyellow}{rwth-yellow!25}
\definecolor{rwth-grass}{cmyk}{.35,0,1,0}\colorlet{rwth-lgrass}{rwth-grass!50}\colorlet{rwth-llgrass}{rwth-grass!25}
\definecolor{rwth-green}{cmyk}{.7,0,1,0}\colorlet{rwth-lgreen}{rwth-green!50}\colorlet{rwth-llgreen}{rwth-green!25}
\definecolor{rwth-cyan}{cmyk}{1,0,.4,0}\colorlet{rwth-lcyan}{rwth-cyan!50}\colorlet{rwth-llcyan}{rwth-cyan!25}
\definecolor{rwth-teal}{cmyk}{1,.3,.5,.3}\colorlet{rwth-lteal}{rwth-teal!50}\colorlet{rwth-llteal}{rwth-teal!25}
\definecolor{rwth-gold}{cmyk}{.35,.46,.7,.35}
\definecolor{rwth-silver}{cmyk}{.39,.31,.32,.14}



% Header i-> inner (bei einseitig links), c -> center, o -> Outer (bei einseitg rechts)
\ihead{SWT WS 2020/21 \\ Gruppe 010 \\\today\\
		Laura Koch, XXXXXX\\
		Marc Ludevid, XXXXXX\\
		Til Mohr, XXXXXX}
\chead{\Large Aufgabenblatt 1}
\ohead{Andrés Montoya, XXXXXX \\
	   Dobri Panayotov, XXXXXX \\
	   Fabian Grob, XXXXXX \\
	   Lennart Holzenkamp, XXXXXX\\
	   Simon Michau, XXXXXX\\
	   Tim Luther, XXXXXX}
	
\cfoot*{\pagemark} % Seitenzahlen unten

\begin{document}
		
	\section*{Aufgabe 1.1}
	\subsection*{a)}
	\begin{table}[h!]
		\begin{tabular}{l|l|l|l|l|l}
		 \textbf{Tätigkeit} 										&A &E &I &T &W \\ 
		 \hline
		 Benutzer der Software schulen								&  &  &  &x &  \\
		 Qualitätssicherung des Pflichtenheftes prüfen				&  &  &  &  &  \\
		 Gesetzliche Rahmenbedingungen prüfen						&x &  &  &  &  \\
		 Konzept und Prototyp einer Benutzeroberfläche erstellen	&  &x &  &  &  \\
		 Entwicklerteam zusammenstellen								&x &  &  &  &  \\
		 Code eines Programmmoduls debuggen							&  &  &x &x &x \\
 		 Zwei Subsysteme verbinden und testen						&  &  &  &  &  \\
 		 Termine und Kosten des Projektes planen					&x &  &  &  &  \\
		 Datenstrukturen festlegen									&  &x &x &  &  \\
		 Vorhandene Altlasten des Kunden analysieren				&x &  &  &  &  \\
		 Schnittstellen von Programmmodulen definieren				&  &  &x &  &  \\
		 Leistung der Entwickler bewerten und belohnen				&  &  &  &  &  \\
		 Software an neue Umgebung anpassen							&  &  &  &  &x \\
		 Kunden eine Rechnung stellen								&x &x &x &x &x \\
		 Test-Eingabedaten für ein Programmmodul ermitteln			&  &  &  &x &  \\ 
		 Strukturmodell des gesamten Softwaresystems entwerfen		&  &x &  &x &  \\ 
		 Dokumentation des Projektablaufes bewerten und archivieren	&  &  &  &x &x \\ 
		 Nach bereits vorhandenen, wiederverwendbaren 				&x &x &  &  &  \\ 
		 	Software-Bibliotheken suchen	&  &  &  &  &  \\ 	
		 Performance-Prognose des Softwaresystems erstellen			&  &x &x &  &  \\ 
		 Programmcode kommentieren									&  &  &x &x &x \\ 
		\end{tabular}
	\end{table}
	*A=Analyse; E=Entwurf; I=Implementierung; T=Test/Integration; W=Wartung\\
	\\\textbf{Begründungen:}\\
	Benutzer der Software schulen: \textcolor{gray}{}\\
	Qualitätssicherung des Pflichtenheftes prüfen\\
	Gesetzliche Rahmenbedingungen prüfen: \textcolor{gray}{Der gesetzliche Rahmen sollte geklärt sein, bevor das Produkt entwickelt wird.}\\
	Konzept und Prototyp einer Benutzeroberfläche erstellen: \textcolor{gray}{Konzept und Prototyp müssen erstellt werden, nachdem in der Analyse bestimmt wurde welche Anforderungen erfüllt werden sollen, aber bevor die Implementierung startet, damit die Entwickler wissen was zu tun ist.} \\
	Entwicklerteam zusammenstellen: \textcolor{gray}{Das Entwicklerteam sollte an das Projekt angepasst werden, bevor dieses startet, aber nachdem geklärt wurde welche Qualifikationen erforderlich sind. Einzelne Teammitglieder können während der Entwicklung noch angepasst werden.}\\
	Code eines Programmmoduls debuggen: \textcolor{gray}{Kann immer passieren, wenn mit konkretem Code gearbeitet wird, vorausgesetzt ein Programm existiert bereits.}\\			
 	Zwei Subsysteme verbinden und testen\\			
 	Termine und Kosten des Projektes planen: \textcolor{gray}{Sollte erledigt sein bevor das Projekt startet.}\\		
	Datenstrukturen festlegen: \textcolor{gray}{Sollte im Entwurf festgelegt werden, damit die Entwickler sie implementieren können. Kann während der Implementierung noch angepasst werden.}\\				
	Vorhandene Altlasten des Kunden analysieren\\			
	Schnittstellen von Programmmodulen definieren\\			
	Leistung der Entwickler bewerten und belohnen\\			
	Software an neue Umgebung anpassen\\		
	Kunden eine Rechnung stellen\\				
	Test-Eingabedaten für ein Programmmodul ermitteln\\	
	Strukturmodell des gesamten Softwaresystems entwerfen\\
	Dokumentation des Projektablaufes bewerten und archivieren\\	
	Nach bereits vorhandenen, wiederverwendbaren Software-Bibliotheken suchen\\
	Performance-Prognose des Softwaresystems erstellen\\
	Programmcode kommentieren\\	
	
	\subsection*{b)}
	\subsection*{c)}
	
	\section*{Aufgabe 1.2}
	
	\section*{Aufgabe 1.3}
	\textbf{funktionale Anforderungen:}
	\begin{itemize}
		\item Bedienbarkeit der Ladestationen
		\item Berechnung der benötigten Lademenge
		\item Äußerung des Ladewunsches
		\item Prüfung ob Fahrzeug angeschlossen ist
		\item Bezahlfunktion
		\item optional Schnellladen, wenn Station eine Schnellladestation ist
	\end{itemize}
	\textbf{nicht-funktionale Anforderungen}
	\begin{itemize}
		\item Bedienung soll über komfortables Interface erfolgen
		\item Überprüfung des Fahrzeugs durch Übertragung von Fahrzeugdaten
		\item Bezahlung nur über Carmpere App oder Smartphone
	\end{itemize}
	
	
\end{document}

