\documentclass[a4paper,11pt]{scrartcl}
\usepackage[a4paper, left=2cm, right=2cm, top=3.5cm, bottom=2cm]{geometry} % kleinere Ränder

%Paket für Header in Koma-Klassen (scrartcl, scrrprt, scrbook, scrlttr2)
\usepackage[headsepline]{scrlayer-scrpage}
% Header groß genug für 3 Zeilen machen
\setlength{\headheight}{3\baselineskip}


% Default Header löschen
\pagestyle{scrheadings}
\clearpairofpagestyles

% nicht kursiv gedruckte header
\setkomafont{pagehead}{\sffamily\upshape}

% Links im Dokuement sowie \url schön machen
\usepackage[colorlinks,pdfpagelabels,pdfstartview = FitH, bookmarksopen = true,bookmarksnumbered = true, linkcolor = black, plainpages = false, hypertexnames = false, citecolor = black]{hyperref}

% Umlaute in der Datei erlauben, auf deutsch umstellen
\usepackage[utf8]{inputenc}
\usepackage[ngerman]{babel}

% Mathesymbole und Ähnliches
\usepackage{amsmath}
\usepackage{mathtools}
\usepackage{amssymb}
\usepackage{microtype}
\newcommand{\NN}{{\mathbb N}}
\newcommand{\RR}{{\mathbb R}}
\newcommand{\QQ}{{\mathbb Q}}
\newcommand{\ZZ}{{\mathbb{Z}}}

% Komplexitätsklassen
\newcommand{\pc}{\ensuremath{{\sf P}}}
\newcommand{\np}{\ensuremath{{\sf NP}}}
\newcommand{\npc}{\ensuremath{{\sf NPC}}}
\newcommand{\pspace}{\ensuremath{{\sf PSPACE}}}
\newcommand{\exptime}{\ensuremath{{\sf EXPTIME}}}
\newcommand{\CClassNP}{\textup{NP}\xspace}
\newcommand{\CClassP}{\textup{P}\xspace}

% Weitere pakete
\usepackage{multicol}
\usepackage{booktabs}

% rwth colors
% colors: blue violet purple carmine red magenta orange yellow grass cyan gold silver
\usepackage{xcolor}
\definecolor{rwth-blue}{cmyk}{1,.5,0,0}\colorlet{rwth-lblue}{rwth-blue!50}\colorlet{rwth-llblue}{rwth-blue!25}
\definecolor{rwth-violet}{cmyk}{.6,.6,0,0}\colorlet{rwth-lviolet}{rwth-violet!50}\colorlet{rwth-llviolet}{rwth-violet!25}
\definecolor{rwth-purple}{cmyk}{.7,1,.35,.15}\colorlet{rwth-lpurple}{rwth-purple!50}\colorlet{rwth-llpurple}{rwth-purple!25}
\definecolor{rwth-carmine}{cmyk}{.25,1,.7,.2}\colorlet{rwth-lcarmine}{rwth-carmine!50}\colorlet{rwth-llcarmine}{rwth-carmine!25}
\definecolor{rwth-red}{cmyk}{.15,1,1,0}\colorlet{rwth-lred}{rwth-red!50}\colorlet{rwth-llred}{rwth-red!25}
\definecolor{rwth-magenta}{cmyk}{0,1,.25,0}\colorlet{rwth-lmagenta}{rwth-magenta!50}\colorlet{rwth-llmagenta}{rwth-magenta!25}
\definecolor{rwth-orange}{cmyk}{0,.4,1,0}\colorlet{rwth-lorange}{rwth-orange!50}\colorlet{rwth-llorange}{rwth-orange!25}
\definecolor{rwth-yellow}{cmyk}{0,0,1,0}\colorlet{rwth-lyellow}{rwth-yellow!50}\colorlet{rwth-llyellow}{rwth-yellow!25}
\definecolor{rwth-grass}{cmyk}{.35,0,1,0}\colorlet{rwth-lgrass}{rwth-grass!50}\colorlet{rwth-llgrass}{rwth-grass!25}
\definecolor{rwth-green}{cmyk}{.7,0,1,0}\colorlet{rwth-lgreen}{rwth-green!50}\colorlet{rwth-llgreen}{rwth-green!25}
\definecolor{rwth-cyan}{cmyk}{1,0,.4,0}\colorlet{rwth-lcyan}{rwth-cyan!50}\colorlet{rwth-llcyan}{rwth-cyan!25}
\definecolor{rwth-teal}{cmyk}{1,.3,.5,.3}\colorlet{rwth-lteal}{rwth-teal!50}\colorlet{rwth-llteal}{rwth-teal!25}
\definecolor{rwth-gold}{cmyk}{.35,.46,.7,.35}
\definecolor{rwth-silver}{cmyk}{.39,.31,.32,.14}\colorlet{rwth-lsilver}{rwth-silver!50}\colorlet{rwth-llsilver}{rwth-silver!25}

% Listings & Code
\usepackage{listings}
\lstdefinestyle{lststyle}{
    backgroundcolor=\color{rwth-llsilver},   
    commentstyle=\color{rwth-green},
    keywordstyle=\color{magenta},
    numberstyle=\tiny\color{rwth-silver},
    stringstyle=\color{rwth-purple},
    basicstyle=\ttfamily\footnotesize,
    breakatwhitespace=false,         
    breaklines=true,                 
    captionpos=b,                    
    keepspaces=true,                 
    numbers=left,                    
    numbersep=5pt,                  
    showspaces=false,                
    showstringspaces=false,
    showtabs=false,                  
    tabsize=2
}
\lstset{style=lststyle}
%define Javascript language
\lstdefinelanguage{JavaScript}{
keywords={typeof, new, true, false, catch, function, return, null, catch, switch, var, if, in, while, do, else, case, break},
keywordstyle=\color{rwth-blue}\bfseries,
ndkeywords={class, export, boolean, throw, implements, import, this, log, abs},
ndkeywordstyle=\color{rwth-red}\bfseries,
identifierstyle=\color{black},
sensitive=false,
comment=[l]{//},
morecomment=[s]{/*}{*/},
commentstyle=\color{rwth-purple}\ttfamily,
stringstyle=\color{rwth-silver}\ttfamily,
morestring=[b]',
morestring=[b]"
}

\lstdefinelanguage{Kotlin}{
keywords={import, fun, var, return, if, while},
keywordstyle=\color{rwth-red}\bfseries,
ndkeywords={println, Int},
ndkeywordstyle=\color{rwth-blue}\bfseries,
identifierstyle=\color{black},
sensitive=false,
comment=[l]{//},
morecomment=[s]{/*}{*/},
commentstyle=\color{rwth-purple}\ttfamily,
stringstyle=\color{rwth-silver}\ttfamily,
morestring=[b]',
morestring=[b]"
}


% Abbildungen
\usepackage{tikz}
\usetikzlibrary{arrows,calc}

% Meistens ist \varphi schöner als \phi, genauso bei \theta
\renewcommand{\phi}{\varphi}
\renewcommand{\theta}{\vartheta}

% Aufzählungen anpassen (alternativ: \arabic, \alph)
\renewcommand{\labelenumi}{(\roman{enumi})}



% Header i-> inner (bei einseitig links), c -> center, o -> Outer (bei einseitg rechts)
\ihead{SWT WS 2020/21 \\ Gruppe 010 \\ Lennart Mesters, 343325 \\
		Laura Koch, 406310\\
		Marc Ludevid, 405401\\
		Til Mohr, 405959}
\chead{\Large Aufgabenblatt 4}
\ohead{Andrés Montoya, 405409 \\
	   Dobromir I. Panayotov, 407763 \\
	   Fabian Grob, 409195 \\
	   Lennart Holzenkamp, 407761\\
	   Simon Michau, 406133\\
	   Tim Luther, 410886}
	
\cfoot*{\pagemark} % Seitenzahlen unten

%%%%%%%%%%%%%%%%%%%%%%%%%%%%%%%%%%%%%%%%%%%%%%%%%%%%%%%%%%%%%%%%%%%%%%%%%%%%%
%%%%%%%%%%%%%%%%%%%%%%%%%%%%%%%%%%%%%%%%%%%%%%%%%%%%%%%%%%%%%%%%%%%%%%%%%%%%%
%%%%%%%%%%%%%%%%%%%%%%%%%%%%%%%%%%%%%%%%%%%%%%%%%%%%%%%%%%%%%%%%%%%%%%%%%%%%%
%%%%%%%%%%%%%%%%%%%%%%%%%%%%%%%%%%%%%%%%%%%%%%%%%%%%%%%%%%%%%%%%%%%%%%%%%%%%%
%%%%%%%%%%%%%%%%%%%%%%%%%%%%%%%%%%%%%%%%%%%%%%%%%%%%%%%%%%%%%%%%%%%%%%%%%%%%%

\begin{document}
		
	\section*{Aufgabe 6.1} 
	\subsection*{a) Java}
	    	\begin{lstlisting}[language=Java]
public class GcdCalculator {

    public static void main(String[] args) {
        //Tests
        GcdCalculator test = new GcdCalculator();
        System.out.println(test.gcd(12,18));
        System.out.println(test.gcd(16,20));
        System.out.println(test.gcd(120,900));
        System.out.println(test.gcd(105,26));
    }

    //Code
    public int gcd(int a, int b) {
        if (a == 0) { return Math.abs(b); }
        if (b == 0) { return Math.abs(a); }
        while (b != 0) {
            int h = a % b;
            a = b;
            b = h;
        }
        return Math.abs(a);
    }
}
            \end{lstlisting}
	
	\subsection*{b) C}
        \begin{lstlisting}[language=C]
#include <stdio.h>
#include <stdlib.h>

int gcd(int a, int b);

int main() {
    //Tests
    printf("%d\n", gcd(12, 18));
    printf("%d\n", gcd(16, 20));
    printf("%d\n", gcd(120, 900));
    printf("%d\n", gcd(105, 26));

    return 0;
}

//Code
int gcd(int a, int b) {
    if (a == 0) { return abs(b); }
    if (b == 0) { return abs(a); }
    while (b != 0) {
        int h = a % b;
        a = b;
        b = h;
    }
    return abs(a);
}
        \end{lstlisting}
    
    \newpage
    \subsection*{c) Python}
        \begin{lstlisting}[language=Python]
#Code
def gcd(a, b):
    if a==0: 
        return abs(b)
    if b==0:
        return abs(a)
    while b!=0:
        h = a % b
        a = b
        b = h
    return abs(a)

#Tests
print(gcd(12,18))
print(gcd(16,20))
print(gcd(120,900))
print(gcd(105,26))
        \end{lstlisting}

    \subsection*{d) JavaScript}
        \begin{lstlisting}[language=JavaScript]
//Code
function gcd(a, b) {
  if (a==0) {
    return Math.abs(b);
  }
  if (b==0) {
    return Math.abs(a)
  }
  while (b!=0){
    var h = a % b;
    a = b;
    b = h;
  }
  return Math.abs(a)
}

//Tests
console.log(gcd(12,18))
console.log(gcd(16,20))
console.log(gcd(120,900))
console.log(gcd(105,26))
        \end{lstlisting}

    \newpage
    \subsection*{e) Go}
        \begin{lstlisting}[language=Go]
package main

import "fmt"

func main() {
    //Tests
    fmt.Println(gcd(12,18))
    fmt.Println(gcd(16,20))
    fmt.Println(gcd(120,900))
    fmt.Println(gcd(105,26))
}

//Code
func gcd(a, b int) int {
    if a==0 {
        return Abs(b)
    }
    if b==0 {
        return Abs(a)
    }
    for {
        if b==0 {
            h := a % b
            a = b
            b = h
        }
    }
    return Abs(a)
}

//Auxiliary function
func Abs(x int) int {
    if x < 0 {
	    return -x
    }
    return x
}
        \end{lstlisting}
        
    \subsection*{f) Kotlin}
        \begin{lstlisting}[language=Kotlin]
import kotlin.math.abs 

fun main() {
    //Tests
    println("${gcd(12,18)}")
    println("${gcd(16,20)}")
    println("${gcd(120,900)}")
    println("${gcd(105,26)}")
}

//Code
fun gcd(a: Int, b: Int): Int {
    var v1 = a
    var v2 = b
    if (a==0) {
        return abs(b)
    }
    if (b==0) {
        return abs(a)
    }
    while (v2!=0){
        var h: Int = v1 % v2
        v1 = v2
        v2 = h
    }
    return abs(v1)
}
        \end{lstlisting}
        
    \section*{Aufgabe 6.2}
\end{document}
