\documentclass[a4paper,11pt]{scrartcl}
\usepackage[a4paper, left=2cm, right=2cm, top=3.5cm, bottom=2cm]{geometry} % kleinere Ränder

%Paket für Header in Koma-Klassen (scrartcl, scrrprt, scrbook, scrlttr2)
\usepackage[headsepline]{scrlayer-scrpage}
% Header groß genug für 3 Zeilen machen
\setlength{\headheight}{3\baselineskip}


% Default Header löschen
\pagestyle{scrheadings}
\clearpairofpagestyles

% nicht kursiv gedruckte header
\setkomafont{pagehead}{\sffamily\upshape}

% Links im Dokuement sowie \url schön machen
\usepackage[colorlinks,pdfpagelabels,pdfstartview = FitH, bookmarksopen = true,bookmarksnumbered = true, linkcolor = black, plainpages = false, hypertexnames = false, citecolor = black]{hyperref}

% Umlaute in der Datei erlauben, auf deutsch umstellen
\usepackage[utf8]{inputenc}
\usepackage[ngerman]{babel}

% Mathesymbole und Ähnliches
\usepackage{amsmath}
\usepackage{mathtools}
\usepackage{amssymb}
\usepackage{microtype}
\newcommand{\NN}{{\mathbb N}}
\newcommand{\RR}{{\mathbb R}}
\newcommand{\QQ}{{\mathbb Q}}
\newcommand{\ZZ}{{\mathbb{Z}}}

% Komplexitätsklassen
\newcommand{\pc}{\ensuremath{{\sf P}}}
\newcommand{\np}{\ensuremath{{\sf NP}}}
\newcommand{\npc}{\ensuremath{{\sf NPC}}}
\newcommand{\pspace}{\ensuremath{{\sf PSPACE}}}
\newcommand{\exptime}{\ensuremath{{\sf EXPTIME}}}
\newcommand{\CClassNP}{\textup{NP}\xspace}
\newcommand{\CClassP}{\textup{P}\xspace}

% Weitere pakete
\usepackage{multicol}
\usepackage{booktabs}

% Abbildungen
\usepackage{tikz}

% Meistens ist \varphi schöner als \phi, genauso bei \theta
\renewcommand{\phi}{\varphi}
\renewcommand{\theta}{\vartheta}

% Aufzählungen anpassen (alternativ: \arabic, \alph)
\renewcommand{\labelenumi}{(\roman{enumi})}

% rwth colors
% colors: blue violet purple carmine red magenta orange yellow grass cyan gold silver
\definecolor{rwth-blue}{cmyk}{1,.5,0,0}\colorlet{rwth-lblue}{rwth-blue!50}\colorlet{rwth-llblue}{rwth-blue!25}
\definecolor{rwth-violet}{cmyk}{.6,.6,0,0}\colorlet{rwth-lviolet}{rwth-violet!50}\colorlet{rwth-llviolet}{rwth-violet!25}
\definecolor{rwth-purple}{cmyk}{.7,1,.35,.15}\colorlet{rwth-lpurple}{rwth-purple!50}\colorlet{rwth-llpurple}{rwth-purple!25}
\definecolor{rwth-carmine}{cmyk}{.25,1,.7,.2}\colorlet{rwth-lcarmine}{rwth-carmine!50}\colorlet{rwth-llcarmine}{rwth-carmine!25}
\definecolor{rwth-red}{cmyk}{.15,1,1,0}\colorlet{rwth-lred}{rwth-red!50}\colorlet{rwth-llred}{rwth-red!25}
\definecolor{rwth-magenta}{cmyk}{0,1,.25,0}\colorlet{rwth-lmagenta}{rwth-magenta!50}\colorlet{rwth-llmagenta}{rwth-magenta!25}
\definecolor{rwth-orange}{cmyk}{0,.4,1,0}\colorlet{rwth-lorange}{rwth-orange!50}\colorlet{rwth-llorange}{rwth-orange!25}
\definecolor{rwth-yellow}{cmyk}{0,0,1,0}\colorlet{rwth-lyellow}{rwth-yellow!50}\colorlet{rwth-llyellow}{rwth-yellow!25}
\definecolor{rwth-grass}{cmyk}{.35,0,1,0}\colorlet{rwth-lgrass}{rwth-grass!50}\colorlet{rwth-llgrass}{rwth-grass!25}
\definecolor{rwth-green}{cmyk}{.7,0,1,0}\colorlet{rwth-lgreen}{rwth-green!50}\colorlet{rwth-llgreen}{rwth-green!25}
\definecolor{rwth-cyan}{cmyk}{1,0,.4,0}\colorlet{rwth-lcyan}{rwth-cyan!50}\colorlet{rwth-llcyan}{rwth-cyan!25}
\definecolor{rwth-teal}{cmyk}{1,.3,.5,.3}\colorlet{rwth-lteal}{rwth-teal!50}\colorlet{rwth-llteal}{rwth-teal!25}
\definecolor{rwth-gold}{cmyk}{.35,.46,.7,.35}
\definecolor{rwth-silver}{cmyk}{.39,.31,.32,.14}

\usetikzlibrary{
  arrows,
  calc,
  shapes,
  arrows,
  shapes.misc,
  shapes.arrows,
  chains,
  matrix,
  positioning,
  scopes,
  decorations.pathmorphing,
  shadows
}

% Header i-> inner (bei einseitig links), c -> center, o -> Outer (bei einseitg rechts)
\ihead{SWT WS 2020/21 \\ Gruppe 010 \\\today\\
		Laura Koch, 406310\\
		Marc Ludevid, 405401\\
		Til Mohr, 405959}
\chead{\Large Aufgabenblatt 1}
\ohead{Andrés Montoya, 405409 \\
	   Dobromir I. Panayotov, 407763 \\
	   Fabian Grob, 409195 \\
	   Lennart Holzenkamp, 407761\\
	   Simon Michau, 406133\\
	   Tim Luther, 410886}
	
\cfoot*{\pagemark} % Seitenzahlen unten

\begin{document}

\tikzset{
  start/.style={circle, fill},
  end/.style={circle, draw, fill = white},
  decision/.style={diamond, black, draw},
  action/.style={rectangle, draw, rounded corners},
  pin/.style={draw,thick,rectangle,minimum height = 0.6em,minimum width=0.6em, node distance=-1pt},
  split/.style={diamond, black, fill}
}

	Gibt leider kein Latex package für Aktivitätsdiagramme, deswegen nicht 100\% korrekt:
	\begin{itemize}
	\item Ausgefüllter Diamand = Aufteilung
	\end{itemize}
		
	\section*{Aufgabe 2.1}

	\begin{tikzpicture}[auto, thick, node distance=1.5cm, >=triangle 45]
	\draw
		node [start](start){}
		
		node [decision, right=of start] (dec1) {}		
		node [decision, below=of dec1] (dec2) {}
		node [action, above right=of dec1] (calc) {Elec. Calc}
		node [action, right=of calc] (test1) {Test 1}
		node [pin, below=of calc] (pin1) {}
		node [pin, below=of test1] (pin2) {}
		node [split, right=of test1] (split1) {}
		node [action, above right=of split1] (prep) {Prepare Charge}
		node [action, below right=of split1] (req) {Request PayMeth}
		node [split, above right=of req] (split2) {}
		node [action, below right=of req] (test2) {Test 2}
		node [decision, below left=of test2] (dec3) {}
		node [action, above left=of dec3] (charge) {Charge}
		node [action, below left=of dec3] (err) {Error}
		node [decision, above left=of err] (dec4) {}
		node [end, left=of dec2](end){}
	;
	
	\draw[->] (start) -- (dec1);
	\draw[->] (dec1) -- node[left]{[Not connected]} ++(0, -1.45cm) -- (dec2);
	\draw[->] (dec1) -- node[above, pos=1pt]{[connected]} ++(0, 1.45cm) -- (calc);
	\draw[->] (calc) -- (test1);
	\draw[->] (pin1) -- node{} ++(0,-0.4cm) -- node{} ++(3.15cm,0) -- (pin2);
	\draw[->] (test1) -- (split1);
	\draw[->] (split1) -- (prep);
	\draw[->] (split1) -- (req);
	\draw[->] (prep) -- (split2);
	\draw[->] (req) -- (split2);
	\draw[->] (split2) -- (test2);
	\draw[->] (test2) -- (dec3);
	\draw[->] (dec3) -- node[above, pos=1pt]{[No error]} ++(0,1.45cm) -- (charge);
	\draw[->] (dec3) -- node[right]{[error]} ++(0,-1.45cm) -- (err);
	\draw[->] (charge) -- (dec4);
	\draw[->] (err) -- (dec4);
	\draw[->] (dec4) -- (dec2);
	\draw[->] (dec2) -- (end);
	
	\end{tikzpicture}
	
	
	
	\section*{Aufgabe 2.1}	
		
\end{document}
