\documentclass[a4paper,11pt]{scrartcl}
\usepackage[a4paper, left=2cm, right=2cm, top=3.5cm, bottom=2cm]{geometry} % kleinere Ränder

%Paket für Header in Koma-Klassen (scrartcl, scrrprt, scrbook, scrlttr2)
\usepackage[headsepline]{scrlayer-scrpage}
% Header groß genug für 3 Zeilen machen
\setlength{\headheight}{3\baselineskip}


% Default Header löschen
\pagestyle{scrheadings}
\clearpairofpagestyles

% nicht kursiv gedruckte header
\setkomafont{pagehead}{\sffamily\upshape}

% Links im Dokuement sowie \url schön machen
\usepackage[colorlinks,pdfpagelabels,pdfstartview = FitH, bookmarksopen = true,bookmarksnumbered = true, linkcolor = black, plainpages = false, hypertexnames = false, citecolor = black]{hyperref}

% Umlaute in der Datei erlauben, auf deutsch umstellen
\usepackage[utf8]{inputenc}
\usepackage[ngerman]{babel}

% Mathesymbole und Ähnliches
\usepackage{amsmath}
\usepackage{mathtools}
\usepackage{amssymb}
\usepackage{microtype}
\newcommand{\NN}{{\mathbb N}}
\newcommand{\RR}{{\mathbb R}}
\newcommand{\QQ}{{\mathbb Q}}
\newcommand{\ZZ}{{\mathbb{Z}}}

% Komplexitätsklassen
\newcommand{\pc}{\ensuremath{{\sf P}}}
\newcommand{\np}{\ensuremath{{\sf NP}}}
\newcommand{\npc}{\ensuremath{{\sf NPC}}}
\newcommand{\pspace}{\ensuremath{{\sf PSPACE}}}
\newcommand{\exptime}{\ensuremath{{\sf EXPTIME}}}
\newcommand{\CClassNP}{\textup{NP}\xspace}
\newcommand{\CClassP}{\textup{P}\xspace}

% Weitere pakete
\usepackage{multicol}
\usepackage{booktabs}
\usepackage{listings}
\lstdefinestyle{lststyle}{  
    commentstyle=\color{rwth-green},
    keywordstyle=\color{rwth-blue},
    stringstyle=\color{rwth-red},
    basicstyle=\ttfamily\footnotesize,
    breakatwhitespace=false,         
    breaklines=true,                 
    captionpos=b,                    
    keepspaces=true,                 
    showspaces=false,                
    showstringspaces=false,
    showtabs=false,                  
    tabsize=2
}
\lstset{style=lststyle}

\lstdefinelanguage{JavaScript}{
keywords={typeof, new, true, false, catch, function, return, null, catch, switch, var, if, in, while, do, else, case, break},
keywordstyle=\color{rwth-blue}\bfseries,
ndkeywords={class, export, boolean, throw, implements, import, this},
ndkeywordstyle=\color{darkgray}\bfseries,
identifierstyle=\color{black},
sensitive=false,
comment=[l]{//},
morecomment=[s]{/*}{*/},
commentstyle=\color{purple}\ttfamily,
stringstyle=\color{rwth-red}\ttfamily,
morestring=[b]',
morestring=[b]"
}

\lstdefinelanguage{Kotlin}{
  comment=[l]{//},
  commentstyle={\color{rwth-gray}\ttfamily},
  emph={delegate, filter, first, firstOrNull, forEach, lazy, map, mapNotNull, println, return@},
  emphstyle={\color{black}},
  identifierstyle=\color{black},
  keywords={abstract, actual, as, as?, break, by, class, companion, continue, data, do, dynamic, else, enum, expect, false, final, for, fun, get, if, import, in, interface, internal, is, null, object, override, package, private, public, return, set, super, suspend, this, throw, true, try, typealias, val, var, vararg, when, where, while},
  keywordstyle={\color{rwth-blue}\bfseries},
  morecomment=[s]{/*}{*/},
  morestring=[b]",
  morestring=[s]{"""*}{*"""},
  ndkeywords={@Deprecated, @JvmField, @JvmName, @JvmOverloads, @JvmStatic, @JvmSynthetic, Array, Byte, Double, Float, Int, Integer, Iterable, Long, Runnable, Short, String},
  ndkeywordstyle={\color{rwth-cyan}\bfseries},
  sensitive=true,
  stringstyle={\color{rwth-red}\ttfamily},
}

% Abbildungen
\usepackage{tikz}
\usetikzlibrary{arrows,calc}

% Meistens ist \varphi schöner als \phi, genauso bei \theta
\renewcommand{\phi}{\varphi}
\renewcommand{\theta}{\vartheta}

% Aufzählungen anpassen (alternativ: \arabic, \alph)
\renewcommand{\labelenumi}{(\roman{enumi})}

% rwth colors
% colors: blue violet purple carmine red magenta orange yellow grass cyan gold silver
\definecolor{rwth-blue}{cmyk}{1,.5,0,0}\colorlet{rwth-lblue}{rwth-blue!50}\colorlet{rwth-llblue}{rwth-blue!25}
\definecolor{rwth-violet}{cmyk}{.6,.6,0,0}\colorlet{rwth-lviolet}{rwth-violet!50}\colorlet{rwth-llviolet}{rwth-violet!25}
\definecolor{rwth-purple}{cmyk}{.7,1,.35,.15}\colorlet{rwth-lpurple}{rwth-purple!50}\colorlet{rwth-llpurple}{rwth-purple!25}
\definecolor{rwth-carmine}{cmyk}{.25,1,.7,.2}\colorlet{rwth-lcarmine}{rwth-carmine!50}\colorlet{rwth-llcarmine}{rwth-carmine!25}
\definecolor{rwth-red}{cmyk}{.15,1,1,0}\colorlet{rwth-lred}{rwth-red!50}\colorlet{rwth-llred}{rwth-red!25}
\definecolor{rwth-magenta}{cmyk}{0,1,.25,0}\colorlet{rwth-lmagenta}{rwth-magenta!50}\colorlet{rwth-llmagenta}{rwth-magenta!25}
\definecolor{rwth-orange}{cmyk}{0,.4,1,0}\colorlet{rwth-lorange}{rwth-orange!50}\colorlet{rwth-llorange}{rwth-orange!25}
\definecolor{rwth-yellow}{cmyk}{0,0,1,0}\colorlet{rwth-lyellow}{rwth-yellow!50}\colorlet{rwth-llyellow}{rwth-yellow!25}
\definecolor{rwth-grass}{cmyk}{.35,0,1,0}\colorlet{rwth-lgrass}{rwth-grass!50}\colorlet{rwth-llgrass}{rwth-grass!25}
\definecolor{rwth-green}{cmyk}{.7,0,1,0}\colorlet{rwth-lgreen}{rwth-green!50}\colorlet{rwth-llgreen}{rwth-green!25}
\definecolor{rwth-cyan}{cmyk}{1,0,.4,0}\colorlet{rwth-lcyan}{rwth-cyan!50}\colorlet{rwth-llcyan}{rwth-cyan!25}
\definecolor{rwth-teal}{cmyk}{1,.3,.5,.3}\colorlet{rwth-lteal}{rwth-teal!50}\colorlet{rwth-llteal}{rwth-teal!25}
\definecolor{rwth-gold}{cmyk}{.35,.46,.7,.35}
\definecolor{rwth-silver}{cmyk}{.39,.31,.32,.14}

\usepackage{pdfpages}


% Header i-> inner (bei einseitig links), c -> center, o -> Outer (bei einseitg rechts)
\ihead{SWT WS 2020/21 \\ Gruppe 010 \\ Lennart Mesters, 343325 \\
		Laura Koch, 406310\\
		Marc Ludevid, 405401\\
		Til Mohr, 405959}
\chead{\Large Aufgabenblatt 6}
\ohead{Andrés Montoya, 405409 \\
	   Dobromir I. Panayotov, 407763 \\
	   Fabian Grob, 409195 \\
	   Lennart Holzenkamp, 407761\\
	   Simon Michau, 406133\\
	   Tim Luther, 410886}
	
\cfoot*{\pagemark} % Seitenzahlen unten

\begin{document}

    \section*{Aufgabe 6.1}
    \renewcommand{\labelenumi}{\alph{enumi})}
		\begin{enumerate}
            \item Java:
\begin{lstlisting} [language=Java]
public class GcdCalculator{

    public static void main(String[] args) {
        System.out.println("Ggt von 12 und 18: " + gcd(12, 18));
        System.out.println("Ggt von 16 und 20: " + gcd(16, 20));
        System.out.println("Ggt von 120 und 900: " + gcd(120, 900));
        System.out.println("Ggt von 105 und 26: " + gcd(105, 26));
    }

    public static int gcd(int a, int b) {
        int h = 0; 

        if( a == 0){
            return Math.abs(b);
        } else if ( b == 0) {
            return Math.abs(a);
        }

        while( b != 0 ) {
            h = a % b;
            a = b;
            b = h;
        }
        return Math.abs(a);
    }
}
\end{lstlisting}
            \item C:
\begin{lstlisting} [language=C]
#include<stdio.h>
#include<stdlib.h>

int gcd(int a, int b) {
    int h;

    if( a == 0 ) {
        return abs(b);
    } else if ( b == 0 ) {
        return abs(a);
    }

    while( b != 0 ) {
        h = a % b;
        a = b;
        b = h;
    }
    return abs(a);
}

int main() {
    printf("Ggt von 12 und 18: %d \n", gcd(12,18));
    printf("Ggt von 16 und 20: %d \n", gcd(16,20));
    printf("Ggt von 120 und 900: %d \n", gcd(120,900));
    printf("Ggt von 105 und 26: %d \n", gcd(105,26));
}
\end{lstlisting}
\newpage
            \item Python:
\begin{lstlisting} [language=Python]
def gcd(a,b):
    h = 0

    if a == 0:
        return abs(b)
    elif b == 0:
        return abs(a)

    while b != 0:
        h = a % b
        a = b
        b = h

    return abs(a)

print("Ggt von 12 und 18: " + str(gcd(12,18)))
print("Ggt von 16 und 20: " + str(gcd(16,20)))
print("Ggt von 120 und 900: " + str(gcd(120,900)))
print("Ggt von 105 und 26: " + str(gcd(105,26)))
\end{lstlisting}
            \item JavaScript
\begin{lstlisting} [language=JavaScript]
function gcd(a, b) {
    h = 0;
    if(a == 0) {
        return Math.abs(b);
    } else if ( b == 0) {
        return Math.abs(a);
    }

    while( b != 0 ) {
        h = a % b;
        a = b;
        b = h;
    }
    return Math.abs(a);
}

console.log("Ggt von 12 und 18: " + gcd(12,18));
console.log("Ggt von 16 und 20: " + gcd(16,20));
console.log("Ggt von 120 und 900: " + gcd(120,900));
console.log("Ggt von 105 und 26: " + gcd(105,26));
\end{lstlisting}
            \item Go
\begin{lstlisting} [language=Go]
package main

import "fmt"

func main() {
	fmt.Println("Ggt von 12 und 18:", gcd(12, 18))
	fmt.Println("Ggt von 16 und 20:", gcd(16, 20))
	fmt.Println("Ggt von 120 und 900:", gcd(120, 900))
	fmt.Println("Ggt von 105 und 26:", gcd(105, 26))
}

func gcd(a, b int) int {

	if a == 0 {
		return abs(b)
	} else if b == 0 {
		return abs(a)
	}

	for b != 0 {
		h := a % b
		a = b
		b = h
	}
	return abs(a)
}

func abs(c int) int {
	if c < 0 {
		return -c
	}
	return c
}
\end{lstlisting}
            \item Kotlin
\begin{lstlisting} [language=Kotlin]
import kotlin.math.abs

fun main() {
    println("Ggt von 12 und 18: " + gcd(12,18))
    println("Ggt von 16 und 20: " + gcd(16,20))
    println("Ggt von 120 und 900: " + gcd(120,900))
    println("Ggt von 105 und 26: " + gcd(105,26))
}

fun gcd(a: Int, b: Int): Int {
    var h: Int
    var x = a
    var y = b
    if ( x == 0) {
        return abs(y)
    } else if( y == 0 ) {
        return abs(x)
    }
    while( y != 0) {
        h = x % y
        x = y
        y = h
    }
    return abs(x)
}
\end{lstlisting}
        \end{enumerate}
    
\newpage

    \section*{Aufgabe 6.2}
    \renewcommand{\labelenumi}{\alph{enumi})}
    \begin{enumerate}
        \item Mit einem Zustandsdiagramm lässt sich darstellen in welchen Zuständen sich ein laufendes System befinden kann und mit
                welchen Ereignissen unter welchen Bedingungen sich diese Zustände verändern können.\\
                Bei einem Sequenzdiagramm hingegen wird eine beispielhaft Folge von Abläufen dargestellt. Dabei wird der Austausch
                zwischen konkreten Objekten beschrieben.\\
                Der Unterschied der beiden Diagramme liegt darin, dass sich mit dem Zustandsdiagramm das Verhalten eines Systems
                darstellen lässt und mit einem Sequenzdiagramm die Interaktion zwischen den einelnen Objekten veranschaulicht wird.
        \item Um ein Softwaresystem zu modellieren, kann kann man Zustands- und Aktivitätsdiagramme kombinieren. Dabei kann man
                Aktionen in einem Zustandsdiagramm, die zu eine Wechesel des Zustands führen durch die Aktionen eines Aktivitätsdiagramms
                modelliert werden.
    \end{enumerate}

\end{document}